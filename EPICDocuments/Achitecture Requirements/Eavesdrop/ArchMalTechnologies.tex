\section{Technologies}


\subsection{Java}
Java is a programming language expressly designed for use in the distributed environment of the Internet. It was designed to have the "look and feel" of the C++ language, but it is simpler to use than C++ and enforces an object-oriented programming model.
Java was used because of its Audio package which provided the tools to manipulate the audio stream and to create the local recordings. 

\subsection{Android SDK}
A software development kit that enables developers to create applications for the Android platform. The Android SDK includes sample projects with source code, development tools, an emulator, and required libraries to build Android applications.
Abdroid SDK was used because a native android appication was developed.
\subsection{TCP/IP}
Transmission Control Protocol/Internet Protocol, TCP/IP is the suite of communications protocols used to connect hosts on the Internet.
This technology was used for communication between the malware application and the server.

\subsection{User Datagram Protocol}
User Datagram Protocol(UDP) is an alternative communications protocol to Transmission Control Protocol (TCP) used primarily for establishing low-latency and loss tolerating connections between applications on the Internet.\\UDP was used because UDP packets are asynchronous unlike TCP where if a packet gets lost it will request the packet again. UDP is required because it will stream live audio and if a packet gets lost it will continue to stream the audio.

\subsection{WebView}
WebView is a browser bundled inside of a mobile application producing what is called a hybrid app. Using a webview allows mobile apps to be built using Web technologies (HTML, JavaScript, CSS, etc.) but still package it as a native app and put it in the app store. 
\\This technology is used because as proof of concept it proves the malware can be embedded in a application in this instance it is hidden behind a WebView browser.