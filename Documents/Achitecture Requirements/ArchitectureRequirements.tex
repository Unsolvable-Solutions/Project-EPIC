\documentclass[a4paper,12pt,titlepage]{article}
\usepackage{graphicx}



\DeclareGraphicsExtensions{.jpg}
\graphicspath{{./Images/}}

\begin{document}
	
\newcommand{\HRule}{\rule{\linewidth}{0.5mm}}
\begin{titlepage}
\begin{center}
\includegraphics[width = 0.3\textwidth]{US_logo.png}~\\[1cm]
\textsc{\LARGE Unsolvable Solutions}\\
Client: CSIR DSSR\\[1.5cm]
\textsc{\Large  Architecturall Requirements}\\[0.5cm]

 \HRule\\[0.4cm]
{ \huge \bfseries  Eavesdropping Protection in Conclave \\[0.4cm] }

 \HRule \\[1.5cm]

\noindent
\begin{minipage}[t]{0.4\textwidth}

	\begin{flushleft} \large
	\emph{Members:}\\
		Edwin Fullard  \\
		Jaco Bezuidenhoudt \\
		Jandre Coetzee\\
		Maret Stoffberg\\
		Ryno Pierce\\
	\end{flushleft}
\end{minipage}%
\begin{minipage}[t]{0.4\textwidth}
\begin{flushright} \large
	\emph{Student Number:} \\
		12048675 \\
		11013878 \\
		 10693077 \\
		 11071762 \\
		 12003922\\
	\end{flushright}
\end{minipage}

\vfill

% Bottom of the page




\end{center}
\end{titlepage}


	
	% Table of content
	\newpage
	\tableofcontents
\newpage \section{Architecture Requirements}
\subsection{Access and Itegration Requirements}
\subsubsection{Human Access Channels}
 	\begin{itemize}
 		\item\b{Via Web Browser}:The user must be able to access the data from the server these standard web browser: Mozilla Firefox, Google Chrome and Microsoft Internet Explorer. It must have a user friendly GUI.
 		\item\b{Via Android device}:This device is used for the malware as well as the protection software.
 		\item\b{Via NFC Nodes and an Edison}:This is used outside the meeting for the access control and the protection application.
		\item\b{The Malware} :The user access the mobile Malware Application via the server and the server can also be accessed via the Mobile Application.
	\end{itemize}
\subsubsection{System Access Channels}
	\begin{itemize}
 		\item\b{Website}: There is interaction with the server via the website.
 		\item\b{Edison}: The Edison send requests to the server and receives responses.
 		\item\b{NFC}: The NFC receive  a email address and password from the mobile application and sends the data then to the Edison.
		\item\b{The Malware}: The Malware server sends request to the Appication and receive an audio stream. 
	\end{itemize}

\subsection{Quality Requirements}
\subsubsection{Scalability}

\begin{itemize}
\item{EPIC} : The server must be able to handle a large amount of users and users must be able to use the system simultaneously. However, the controlled access to a meeting, via the Gateway and the Node will only have a limit of one gateway per meeting and at most three nodes per meeting.
\item{Malware} : The server can only handle one live stream recording at a time.
\end{itemize}
\subsubsection{Performance Requirements}
The system does not have specific performance requirements, but the application to gateway operations should respond within less than 1 second, because it may delay the meeting unnecessarily. The server query operations may process up to 5 seconds long.

\subsubsection{Maintainability}
The system must be easily maintainable. The application and server may be updated in time, to fit the latest technologies. 

\subsubsection{Reliability and Availability}
The application must be available from the Google play store, and the access to the web server can be found online at projectepic.info.

\subsubsection{Auditability}
The system log the entrance and exit of all meetings. For each meeting the log contains the users allowed, the users that attended, their entrance and exit times as well as the initial time and place of the meeting.

\subsubsection{Security}
The NFC components should not be hackable. 

\subsubsection{Monitorability}
The response of the Nodes is visible with the LED light showing the permission.

\subsubsection{Testability}
All the components must be tested separate at first, unit testing, and after completion the whole system must be tested.
\\ For each service the pre and post conditios must be met.

\subsubsection{Usability}
All the components must be intuitive and easy to use. Any user that is Android literate and computer literate must be able to use the system. The physical components must be labelled accordingly. 
\\ The User Manual may be used for more information on any of the components. 

\subsubsection{Integrability}
The different components of the system should work together and the system must also be able to handle future additions to it.\\

\subsection{Architecture constraints}
	\begin{itemize}
		\item The Mobile application can only be used on Andriod version 4.4 and 5.1.
		\item The Node can only communite with one device at a time.
		\item The Node can only work if the specific Mobile Application is installed.
		\item The Edison  can only be connected to three Nodes maximum at a time.
		\item The Malware application can only be used on Android version 4.4 and higher.
		\item The Malware server has a minimum Java version 7.
		\item The Malware may run on any operating system that supports Virtual Machine.

	\end{itemize}
\newpage	\section{Architectural patterns or styles}
\begin{itemize}
\item Thread pooling :to improve scalability. \\It is used by the recording handler of the Malware Application and to create multiple connections to different client phones. The Edison is connected to more than one node to improve scalability( thus physical threads).
\item Clustering: to improve scalability and reliability\\
\item Interception: for security and auditability\\ 
\item Run-time lookup:  for pluggability and flexibility.\\
\item Queuing: reliability and scalability.\\
\end{itemize}


\newpage\section{Architectural tactics or strategies}
\begin{itemize}
\item Thread pooling :to improve scalability. //It is used by the recording handler of the Malware Application and to create multiple connections to different client phones. The Edison is connected to more than one node to improve scalability( thus physical threads).
\item Clustering: to improve scalability and reliability.//
\item Interception: for security and auditability//
\item Run-time lookup:  for pluggability and flexibility.//
\item Queuing: reliability and scalability.//
\end{itemize}

\newpage	\section{Use of reference architectures and frameworks}


\newpage	\section{Access and integration channels}
\subsection{EPIC}
	\begin{itemize}
		\item The user interact and have access to the system via the website ( www.project-epic.info ). The user may register as a user, log in, create a meeting,view all upcomming and past meetings and view,delete and edit his own account. The website then sends the requests for these functions to the server. If the user is an administrative user, they may additionally add more administrators, remove meetings, view all user details.
\item The Edison sends a request to the server for the attendant list by sending the meeting identification. The Edison then have the list of attendees and updates the list every hour or on a request.
\item The Nodes communicate with the Edison via serial port. The Nodes send the email address and password that they received to the Edison to confirm if the user may enter the meeting. The Edison respond with a true or false, if the user may enter or not. The Nodes neopixels will then turn red if access is denied and green if access is granted.
\item The Mobile Application communicates with the Node via NFC. It sends the email address and password to the Node and receive a response stating if the user may enter the meeting or not. If access is granted the phones screen turn green and all communication is shut down, otherwise the screen just turns red.
\item The  user have to install the Mobile Application. If the user wants to scan the phone, he has to open the application before scanning. The first time the user opens the application he has to fill out his details. 

	\end{itemize}
\subsection{Malware}
	\begin{itemize}
		\item The user may acces the server to control the application, the server then communicates with the application via TCP
	\end{itemize}
\newpage	\section{Technologies}
\begin{itemize}
\item NFC
\item NodeJS
\item Arduino
\item MongoDB
\item Passport
\item Java
\item Android SDK
\item TCP
\item UDP
\item HCE
\item INDEF
\item JavaScript
\item AngularJS
\end{itemize}

\end{document}